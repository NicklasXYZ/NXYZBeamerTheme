\documentclass[aspectratio=169]{beamer}

% Packages used:
\usepackage{tikz}
\usepackage{pgfplots}
\usepackage{float}
\usepackage{amsmath}
\usepackage[utf8]{inputenc}
\usepackage{listings}
\usepackage{url}
\usepackage{hyperref}

% Misc. settings:
\pgfplotsset{compat=newest}

% lstlistings style:
\lstset{
    basicstyle=\ttfamily\small,
    numbers=left,
    numberstyle=\small\color{base-color-one},
    xleftmargin=0pt,
    xrightmargin=0pt,
    stepnumber=1,
    showstringspaces=false,
    showtabs=false,
    showspaces=false,
    columns=fullflexible,
}

% External link icon:
\newcommand{\ExternalLink}{%
    \tikz[x=1.2ex, y=1.2ex, baseline=-0.05ex]{% 
        \begin{scope}[x=1ex, y=1ex]
            \clip (-0.1,-0.1)
                --++ (-0, 1.2)
                --++ (0.6, 0)
                --++ (0, -0.6)
                --++ (0.6, 0)
                --++ (0, -1);
            \path[draw, 
                line width = 0.5,
                rounded corners=0.5]
                (0,0) rectangle (1,1);
        \end{scope}%
        \path[draw, line width = 0.5] (0.5, 0.5) 
            -- (1, 1);
        \path[draw, line width = 0.5] (0.6, 1) 
            -- (1, 1) -- (1, 0.6);
    }%
}%

% Specify the theme to use:
\usetheme{NXYZ}          % (Default) red theme 
% \usetheme[red]{NXYZ}   % Red theme 
% \usetheme[black]{NXYZ} % Black theme
% \usetheme[blue]{NXYZ}  % Blue theme
% \usetheme[green]{NXYZ} % Blue theme

% Doc. settings
\title{Typesetting with \LaTeX}
\subtitle{Example slides}
\author{Name Nameson}
\institute{Institute}
\date{April, 2020}

% Document start
\begin{document}

    % Title page:
    \begin{frame}
        \titlepage
    \end{frame}

    % Table of contents
    \begin{frame}{Overview}
        \tableofcontents
    \end{frame}

    \section{Creating a Basic Document}

    \subsection{A Simple Example}
    
    \begin{frame}[fragile]{Creating a Basic Document}
        \framesubtitle{A minimal working example}
        Write a \texttt{.tex} file containing:
\begin{lstlisting}
\documentclass[a4paper, 11pt]{article}
% Preamble
\begin{document}
  % The body of the document
  A simple \LaTeX document.
\end{document}
\end{lstlisting}
    \end{frame}

    \section{Main Components}

    \subsection{Declarations, Enviroments \& Other Things}

    \begin{frame}{Declarations \& Enviroments}
        \begin{block}{Declarations...}
            \begin{itemize}
                \item Are stated once and take effect until further notice
                \item Changes the formatting globally
            \end{itemize}
        Example: {\small\textbackslash\texttt{documentclass}}
        \end{block}
        \begin{block}{Enviroments...}
            \begin{itemize}
                \item Are delimited by matching {\small\textbackslash \texttt{begin}} and {\small\textbackslash\texttt{end}} declarations
                \item Changes the formatting locally
            \end{itemize}
        Example: {\small\textbackslash\texttt{begin\{document\} ...}}  {\small\textbackslash\texttt{end\{document\}}}
        \end{block}
    \end{frame}

    \begin{frame}{Arguments}
        \begin{block}{Required Arguments}
            \begin{itemize}
                \item Can not be left out
                \item Are contained in curly braces: {\small\texttt{\{ \textit{required argument} \}}}
            \end{itemize}
            Example: {\small\textbackslash\texttt{documentclass\{article\}}}
        \end{block}
        \begin{block}{Optional Arguments}
            \begin{itemize}
                \item Can be left out
                \item Are contained in square brackets: {\small\texttt{[ \textit{optional argument} ]}}
            \end{itemize}
            Example: {\small\textbackslash\texttt{documentclass[a4paper, 11pt]\{article\}}}
        \end{block}
    \end{frame}

    \begin{frame}[fragile]{Special Characters}
        \begin{itemize}
            \item {\small\texttt{\textbackslash}} ~ backslash precedes all \LaTeX\ commands.
            \item {\small\texttt{\{\}}} ~ curly braces group or seperate commands
            \item {\small\texttt{\$}} ~ dollor sign begin or end inline math mode
            \item {\small\texttt{\^{}}} ~ caret is used in math mode for superscripts
            \item {\small\texttt{\_}} ~ underscore is used in math mode for subscripts
            \item {\small\texttt{\%}} ~ percentage sign starts a comment
            \item {\small\texttt{\~}} ~ tilde defines an unbreakable space
            \item {\small\texttt{\&}} ~ ampersand is used to align text in certain enviroments
        \end{itemize}
    \end{frame}

    \begin{frame}[fragile]{Front Page}
        A front page with a table of contents, author and date can be created by including the declarations shown below:
\begin{lstlisting}
% Preamble
\title{\LaTeX\ Document Title}
\author{Author}
\date{\today}

\begin{document}
  \maketitle
  \tableofcontents 
  % The body of the document
  A simple \LaTeX\ document.
\end{document}
\end{lstlisting}
    \end{frame}

    \begin{frame}{Sectioning}
        In a simple document it is usually sufficient to simply use: 
        \begin{itemize}
            \item Level 1: {\small\texttt{\textbackslash section}} ~
            \item Level 2: {\small\texttt{\textbackslash subsection}} ~
        \end{itemize}
        Several other levels of sectioning can be obtained by using:
        \begin{itemize}
            \item Level 0: {\small\texttt{\textbackslash chapter\{\textit{ chapter name }\}}} ~
            \item Level 3: {\small\texttt{\textbackslash subsubsection\{\textit{ subsubsection name }\}}} ~
            \item Level 4: {\small\texttt{\textbackslash paragraph\{\textit{ paragraph name }\}}} ~
            \item Level 5: {\small\texttt{\textbackslash subparagraph\{\textit{ subpragraph name }\}}}
        \end{itemize}
        The level of sectioning that is possible usually depend on the documentclass.
    \end{frame}

    \begin{frame}{Font Faces}
        \begin{itemize}
            \item {\small\textbackslash\texttt{emph\{Text\}}}~$\rightarrow$~\emph{Text}
            \item {\small\textbackslash\texttt{textbf\{Text\}}}~$\rightarrow$~\textbf{Text}
            \item {\small\textbackslash\texttt{texttt\{Text\}}}~$\rightarrow$~\texttt{Text}
            \item {\small\textbackslash\texttt{textrm\{Text\}}}~$\rightarrow$~\textrm{Text}
            \item {\small\textbackslash\texttt{textsf\{Text\}}}~$\rightarrow$~\textsf{Text}
            \item {\small\textbackslash\texttt{textsc\{Text\}}}~$\rightarrow$~\textsc{Text}.	
        \end{itemize}
    \end{frame}
        
    \begin{frame}{Font Sizes}
        \begin{itemize}
            \item {\small\textbackslash\texttt{tiny\{Text\}}}~$\rightarrow$~{\tiny{Text}}
            \item {\small\textbackslash\texttt{scriptsize\{Text\}}}~$\rightarrow$~{\scriptsize{Text}}
            \item {\small\textbackslash\texttt{footnotesize\{Text\}}}~$\rightarrow$~{\texttt{Text}}
            \item {\small\textbackslash\texttt{small\{Text\}}}~$\rightarrow$~{\small{Text}}
            \item {\small\textbackslash\texttt{normalsize\{Text\}}}~$\rightarrow$~{\normalsize{Text}}
            \item {\small\textbackslash\texttt{large\{Text\}}}~$\rightarrow$~{\large{Text}}
            \item {\small\textbackslash\texttt{Large\{Text\}}}~$\rightarrow$~{\Large{Text}}
            \item {\small\textbackslash\texttt{LARGE\{Text\}}}~$\rightarrow$~{\LARGE{Text}}
            \item {\small\textbackslash\texttt{huge\{Text\}}}~$\rightarrow$~{\huge{Text}}
            \item {\small\textbackslash\texttt{Huge\{Text\}}}~$\rightarrow$~{\Huge{Text}}
        \end{itemize}	
    \end{frame}
        
    \begin{frame}[fragile]{Text Alignment}
        \begin{columns}
            \begin{column}{0.45\textwidth}
\begin{lstlisting}
\begin{flushleft}
  Text
\end{flushleft}
\end{lstlisting}
\vspace{0.5cm}
\begin{lstlisting}
\begin{center}
  Text
\end{center}
\end{lstlisting}
\vspace{0.5cm}	
\begin{lstlisting}
\begin{flushright}
  Text
\end{flushright}
\end{lstlisting}
        \end{column}
        \begin{column}{0.45\textwidth}
            \begin{flushleft}
                Text
            \end{flushleft}
            \vspace{1.0cm}
            \begin{center}
                Text
            \end{center}
            \vspace{1.0cm}
            \begin{flushright}
                Text
            \end{flushright}
            \vspace{0.20cm}
            \end{column}
        \end{columns}
    \end{frame}
    
    \begin{frame}{Spacing}
        \begin{block}{Horizontal Spacing $(\leftrightarrow)$}
            \begin{itemize}
                \item {\small\texttt{\~}} ~ unbreakable space
                \item {\small\textbackslash\texttt{hspace\{\}}} ~ user-specifiable spacing
                \item {\small\textbackslash\texttt{quad}}
                \item {\small\textbackslash\texttt{qquad}} 
                \item {\small\textbackslash\texttt{,}}
                \item {\small\textbackslash\texttt{;}}
            \end{itemize}
        \end{block}
        \begin{block}{Vertical Spacing $(\updownarrow)$}
            \begin{itemize}
                \item {\small\textbackslash\texttt{vspace\{\}}} ~
            \end{itemize}
        \end{block}
    \end{frame}
    
    \begin{frame}{Line \& Page Breaking }
        \begin{itemize}
            \item {\small\texttt{\textbackslash\textbackslash}} ~ starts a new paragraph
            \item {\small\texttt{\textbackslash\textbackslash$*$}} ~ starts a new line, but not a new paragraph
            \item {\small\texttt{\textbackslash cleardoublepage}} ~ flushes all material and start a new page
            \item {\small\texttt{\textbackslash clearpage}} ~ flushes all material and starts a new page
            \item {\small\texttt{\textbackslash linebreak}} ~ allow linebreaking here
            \item {\small\texttt{\textbackslash newline}} ~ requests a new line
            \item {\small\texttt{\textbackslash newpage}} ~ requests a new page
            \item {\small\texttt{\textbackslash nolinebreak}} ~ no line break should happen here
            \item {\small\texttt{\textbackslash nopagebreak}} ~ no page break should happen here
            \item {\small\texttt{\textbackslash pagebreak}} ~ encourage page break
        \end{itemize}
    \end{frame}
        
    \begin{frame}[fragile]{List Enviroments}
        \begin{columns}
            \begin{column}{0.30\textwidth}
\begin{lstlisting}
\begin{itemize}
  \item Text
  \item Text
\end{itemize}
\end{lstlisting}\vspace{0.5cm}
                \texttt{itemize:}
                \begin{itemize}
                    \item Text
                    \item Text
                \end{itemize}
            \end{column}
        \begin{column}{0.30\textwidth}
\begin{lstlisting}
\begin{enumerate}
  \item Text
  \item Text
\end{enumerate}
\end{lstlisting}\vspace{0.5cm}
                \texttt{enumerate:}
                \begin{enumerate}
                    \item Text
                    \item Text
                \end{enumerate}
            \end{column}
        \begin{column}{0.30\textwidth}
\begin{lstlisting}
\begin{description}
  \item[P1] Text
  \item[P2] Text
\end{description}
\end{lstlisting}\vspace{0.5cm}
                \texttt{description:}
                \begin{description}[]
                    \item[P1] Text
                    \item[P2] Text 
                \end{description}
            \end{column}
        \end{columns}
    \end{frame}

    \begin{frame}{Packages}
        \begin{block}{Packages...}
            \begin{itemize}
                \item extend the default functionality and customizability of \LaTeX.
                \item are declared in the premable
                \item are imported using the command:\\ 
                \begin{center}
                    \texttt{\textbackslash usepackage\{\textit{package name}\}}
                \end{center}
            \end{itemize}
        \end{block}
        \begin{block}{Some packages:}
            {\small\texttt{amsmath}, \texttt{amssymb}, \texttt{amsthm}, \texttt{tikz}, \texttt{pgfplots}, \texttt{lstlistings}, \texttt{url}, \texttt{hyperref}, \texttt{xcolor}, \texttt{float}, \texttt{fancyhdr}, \texttt{epstopdf} and many more.}
        \end{block}
    \end{frame}

    \subsection{Typesetting Mathematics}

    \begin{frame}[fragile]{Typesetting Math}
        For mathematical typesetting we include the \texttt{amsmath}, \texttt{amssymb}, \texttt{amsthm} packages: 
\begin{lstlisting}
% Preamble
\usepackage{amsmath}
\usepackage{amssymb}
\usepackage{amsthm}
\end{lstlisting}
    These packages contain various enviroments, symbols and commands, that are useful when typesetting mathematics.
    \end{frame}

    \begin{frame}{Symbols}
        A huge number of symbols are available in \LaTeX. It is thus very convenient to know a number of resources where the commands for these symbols can be found.
        \begin{block}{Symbol List \& Detexify}
            Having trouble finding a special symbol? Try:
            \begin{itemize}
                \item Detexify \href{http://detexify.kirelabs.org/classify.html}{\ExternalLink}
                \item The comprehensive \LaTeX\ symbol list \href{http://mirror.utexas.edu/ctan/info/symbols/comprehensive/symbols-a4.pdf}{\ExternalLink}
            \end{itemize}
        \end{block}	
    \end{frame}
    
    \begin{frame}{Functions}
        Some common functions can easily be written in \LaTeX:
        \begin{itemize}
            \item {\small\textbackslash\texttt{sin}} $\rightarrow$ $\sin$
            \item {\small\textbackslash\texttt{cos}} $\rightarrow$ $\cos$
            \item {\small\textbackslash\texttt{ln}} $\rightarrow$ $\ln$
            \item {\small\textbackslash\texttt{log}} $\rightarrow$ $\log$
            \item {\small\textbackslash\texttt{exp}} $\rightarrow$ $\exp$
            \item {\small\textbackslash\texttt{mod}} $\rightarrow$ $\mod$
            \item {\small\textbackslash\texttt{sqrt\{x\}}} $\rightarrow$ $\sqrt{x}$
        \end{itemize}
        These functions will already have the desired formatting in the misc. math enviroments. 
    \end{frame}

    \begin{frame}[fragile]{Equation Enviroments}
        Several different equation enviroments exist:
        \begin{itemize}
            \item {\small\texttt{equation}}
            \item {\small\texttt{align}}
            \item {\small\texttt{gather}}
            \item {\small\texttt{multiline}}
            \item ... and so on
        \end{itemize}
        It is usually sufficient to simply stick with one or two of these and use additional enviroments to obtain the desired formatting.
    \end{frame}

    \begin{frame}[fragile]{Equation Enviroments}
        \begin{columns}
            \begin{column}{0.45\textwidth}
                With numbering:
\begin{lstlisting}
\begin{align}
  2x + 3y = 6
\end{align}
\end{lstlisting}\vspace{0.5cm}
                \texttt{align:}
                \begin{align}
                    2x + 3y = 6
                \end{align}
            \end{column}
            \begin{column}{0.45\textwidth}
\begin{lstlisting}
\begin{equation}
  2x + 3y = 6 
\end{equation}
\end{lstlisting}\vspace{0.5cm}
                \texttt{equation:}
                \begin{equation}
                  2x + 3y = 6
                \end{equation}
            \end{column}
        \end{columns}
    \end{frame}
    
    \begin{frame}[fragile]{Equation Enviroments}
        \begin{columns}
            \begin{column}{0.45\textwidth}
                Without numbering:
\begin{lstlisting}
\begin{align*}
  2x + 3y = 6
\end{align*}
\end{lstlisting}\vspace{0.5cm}
                \texttt{align:}
                \begin{align*}
                    2x + 3y = 6
                \end{align*}
            \end{column}
            \begin{column}{0.45\textwidth}
\begin{lstlisting}
\begin{equation*}
  2x + 3y = 6 
\end{equation*}
\end{lstlisting}\vspace{0.5cm}
                \texttt{equation:}
                \begin{equation*}
                  2x + 3y = 6
                \end{equation*}
            \end{column}
        \end{columns}
    \end{frame}

    \begin{frame}[fragile]{Aligning Several Equations}
        \begin{columns}
            \begin{column}{0.45\textwidth}
                The following \LaTeX\ code: 
\begin{lstlisting}
\begin{align*}
  2x + 3y  &= 6 \\
  3x + 4y  &= 8 \\
  4x + 5y &= 9
\end{align*}
\end{lstlisting}\vspace{0.5cm}
                Produces:
                \begin{align*}
                    2x + 3y &= 6\\
                    3x + 4y &= 8\\
                    4x + 5y &= 9
                \end{align*}
            \end{column}
            \begin{column}{0.45\textwidth}
                The following \LaTeX\ code: 
\begin{lstlisting}
\begin{align*}
  2x + 3y  &= 6,  & a = 1\\
  3x + 4y  &= 8, & b = 2\\
  4x + 5y &= 9, & c = 3
\end{align*}
\end{lstlisting}\vspace{0.5cm}
                Produces:
                \begin{align*}
                    2x + 3y &= 6, & a = 1\\
                    3x + 4y &= 8, & b = 2\\
                    4x + 5y &= 9, & c = 3
                \end{align*}
            \end{column}
        \end{columns}
    \end{frame}

    \begin{frame}[fragile]{Subscripts \& Superscripts}
        Caret {\small\texttt{\^{}}} and underscore {\small\texttt{\_}} are used for superscripts and subscripts. Two examples of their use:
        \begin{columns}
            \begin{column}{0.45\textwidth}
\begin{lstlisting}
\begin{align*}
  \prod_{i=1}^{n} i
\end{align*}
\end{lstlisting}\vspace{0.5cm}
                Produces:
                \begin{align*}
                    \prod_{i = 1}^{n} i
                \end{align*}
            \end{column}
            \begin{column}{0.45\textwidth}
\begin{lstlisting}
\begin{align*}
  \sum_{i = 0}^{\infty} x^i
\end{align*}
\end{lstlisting}\vspace{0.5cm}
                Produces:
                \begin{align*}
                    \sum_{i = 0}^{\infty} x^i
                \end{align*}
            \end{column}
        \end{columns}
    \end{frame}

    \begin{frame}[fragile]{Matrix Enviroments}
        \begin{columns}
            \begin{column}{0.30\textwidth}
\begin{lstlisting}
\begin{Vmatrix}
  a & b \\
  c & d 	
\end{Vmatrix}
\end{lstlisting}\vspace{0.5cm}
            \begin{align*}
            \texttt{Vmatrix:}&
            \begin{Vmatrix}
              a & b \\
              c & d 	
            \end{Vmatrix}
            \end{align*}
            \end{column}
            \begin{column}{0.30\textwidth}
\begin{lstlisting}
\begin{vmatrix}
  a & b \\
  c & d 	
\end{vmatrix}		
\end{lstlisting}\vspace{0.5cm}
            \begin{align*}
            \texttt{vmatrix:}&
            \begin{vmatrix}
              a & b \\
              c & d 	
            \end{vmatrix}
            \end{align*}
            \end{column}
            \begin{column}{0.30\textwidth}
\begin{lstlisting}
\begin{Bmatrix}
  a & b \\
  c & d 	
\end{Bmatrix}
\end{lstlisting}\vspace{0.5cm}
            \begin{align*}
            \texttt{Bmatrix:}&
            \begin{Bmatrix}
                a & b \\
                c & d 	
            \end{Bmatrix}
            \end{align*}
            \end{column}
        \end{columns}
    \end{frame}

    \begin{frame}[fragile]{Matrix Enviroments}
        \begin{columns}
            \begin{column}{0.30\textwidth}
\begin{lstlisting}
\begin{Bmatrix}
  a & b \\
  c & d 	
\end{Bmatrix}
\end{lstlisting}\vspace{0.5cm}
                \begin{align*}
                    \texttt{bmatrix:}& 
                    \begin{bmatrix}
                        a & b \\
                        c & d 	
                    \end{bmatrix}			
                \end{align*}
                \end{column}
                \begin{column}{0.30\textwidth}
\begin{lstlisting}
\begin{matrix}
  a & b \\
  c & d 	
\end{matrix}
\end{lstlisting}\vspace{0.5cm} 
                \begin{align*}
                    \hspace{-0.40cm}\texttt{matrix:}\hspace{0.10cm}& 
                    \begin{matrix}
                        a & b \\
                        c & d 	
                    \end{matrix}			
                \end{align*}
                \end{column}
                \begin{column}{0.30\textwidth}
\begin{lstlisting}
\begin{pmatrix}
  a & b \\
  c & d 	
\end{pmatrix}
\end{lstlisting}\vspace{0.5cm}
                \begin{align*}
                    \texttt{pmatrix:}& 
                    \begin{pmatrix}
                        a & b \\
                        c & d 	
                    \end{pmatrix}
                \end{align*}
                \end{column}
            \end{columns}
    \end{frame}

    \subsection{Definitions, Theorems \& More...}

    \begin{frame}[fragile]{Definitions, Theorems \& More...}
        ... Can be created by including the {\small\texttt{amsthm}} package and defining in the preamble:
\begin{lstlisting}
% Preamble
\newtheorem{theorem}{Theorem}
\end{lstlisting}
A theorem and corresponding proof can then be written:
\begin{lstlisting}
% The body of the document
\begin{theorem}
  % Theorem content
\end{theorem}
\begin{proof}
  % Proof content
\end{proof}
\end{lstlisting} 
    \end{frame}

    \subsection{Labeling \& Referencing}

    \begin{frame}[fragile]{Labeling \& Referencing}
        \begin{block}{Labeling}
            A label can be assigned to a certain enviroment:
            \begin{center}
                {\small\textbackslash \texttt{label\{\textit{ label name }\}}}
            \end{center}
        \end{block}
        \begin{block}{Referencing}
            The enviroment can then be referenced, by using the label:
            \begin{center}
                {\small\textbackslash \texttt{ref\{\textit{ label name }\}}}
            \end{center}
        \end{block}
    \end{frame}
    
    \begin{frame}[fragile]{Labeling \& Referencing Example}	
        The following \LaTeX\ code:
\begin{lstlisting}
\begin{align} \label{eq:SimpleEq}
  a_1 = b_1 + c_1		
\end{align}
We reference Eqn.~\ref{eq:SimpleEq}.
\end{lstlisting}\vspace{0.5cm}
        Produces:
        \begin{align} \label{eq:SimpleEq}
            a_1 = b_1 + c_1		
        \end{align}
        We reference Eqn.~\ref{eq:SimpleEq}.
    \end{frame}

    \subsection{Managing a Bibliography}

    \begin{frame}[fragile]{Creating a Bibliography}
        A simple bibliography entry in the {\small\texttt{thebibliography}} enviroment:
\begin{lstlisting}
\begin{thebibliography}{0}
  \bibitem{ItemName}
  Author One and Author Two,
  \textit{Article Title}.
  Journal,
  Year,
  Vol.,
  pp. 0--9999.
\end{thebibliography}
\end{lstlisting}
    \end{frame}

    \begin{frame}[fragile]{Citations}
        After a bibliography has been created the entries can be cited:
\begin{lstlisting}
  \cite{ItemName} % Cite article
  \cite[p.~9999]{ItemName} % Cite article page
  \cite[pp.~0--9999]{ItemName} % Cite article page range
\end{lstlisting}
    \end{frame}

    \section{Additionl Resources}

    \begin{frame}[fragile]{Additional Resources}
        Misc. introductory material, useful packages and other things:
        \begin{itemize}
            \item \LaTeX\ on Wikibooks~\href{https://en.wikibooks.org/wiki/LaTeX}{\ExternalLink}
            \item The Not So Short Introduction to \LaTeX~\href{https://www.ctan.org/tex-archive/info/lshort/english/}{\ExternalLink}
            \item \LaTeX\ on AoPS~\href{https://artofproblemsolving.com/wiki/index.php/LaTeX:LaTeX_on_AoPS}{\ExternalLink}
            \item TikZ (package)~\href{https://www.ctan.org/pkg/pgf}{\ExternalLink}
            \item PGFPlots (package)~\href{https://ctan.org/pkg/pgfplots}{\ExternalLink}
        \end{itemize}
    \end{frame}

    \section{Contact Information}

    \begin{frame}[fragile]{Contact Information}
        \begin{description}
            \item[Email]  : \href{mailto:username@domain.com}{username@domain.com}
        \end{description}
    \end{frame}

% Document end...
\end{document}